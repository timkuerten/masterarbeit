\section{Conclusion}
\label{sec:conclusion}
%==============================================================================
\blindtext



\begin{theorem}[Lorenz-Langmann]
Dieses Theorem hat eine witzige Vermutung, die hier nicht näher ausgeführt werden soll.
Es folgt der Beweis mit einem Beispiel für die mehrzeilige IEEE Equationarray Umgebung:
\begin{IEEEproof}[Beweis]\label{example:theorem.multiline}
Der Beweis ist sehr simpel. Und er eigentlich ist es auch keiner.
\begin{IEEEeqnarray}{rCl}
a	& = 	& b + c				\\
	& = 	& d + e + f + g + h
	+ i + j + k \nonumber		\\
	&		& +\> l + m + n + o		\\
	\noalign{Mittels des \texttt{noalign}-Befehlt lässt sich leicht ein Text innerhalb der IEEEeqnarray-Umgebung einfügen}
	& = 	& p + q + r + s
\end{IEEEeqnarray}
\end{IEEEproof}
\end{theorem}

\begin{definition}[$\mathfrak{t}$-Funktion]
Die Funktion $\mathfrak{t}_W:\mathbb{R}\rightarrow W$ ist definiert durch:
\[
\mathfrak{t}_W(x):=\max\left\lbrace k \in W \right\rbrace
\]
Macht keinen großen Sinn, oder?
\end{definition}

\Blindtext
\todo{Das steht total losgelöst vom Text, bitte ändern!}
 Es folgt ein tolles Diagramm:
 \label{example:diagram}
\[
\begin{tikzcd}[column sep=huge]
\mbox{} & A \arrow{r}{f} \arrow{d}{a} & B \arrow{r}{g} \arrow{d}{b} & C \arrow{r} \arrow{d}{c} & 0 \\
0 \arrow{r} & A' \arrow{r}{f'} & B' \arrow{r}{g'} & C' \end{tikzcd} \]
\Blindtext
